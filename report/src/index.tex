
% \section*{Introduction}




% % - Summary of the project background and motivation
% % - Introduce your project’s problem
% % - About the problem, not the project
% % - Why are you working on this problem?

% \section*{Team Members \& Bios}


% \section*{Background and Related Work}
% % Context for the project problem
% % • Which field of computer science does the project problem belong to?
% % • What have other done to solve the project problem?
% % • Related work includes
% %   • Books, research articles, open source repos, white / tech papers
% %   • List citations for related work in references section and cite them
% %   • IEEE citation and referencing style
% %   • PLEASE CITE
% % • Describe which skills you need to complete this project



% \section*{Project Overview}
% % Summarize your project
% % • What will you do to solve / address the project problem?
% % • Describe goals and desired outcomes
% % • Shows understanding of the project objectives and outcomes
% % • Bulk of the document (1 page)




% \section*{Client/Stakeholder Information}
% % Identify your clients and stakeholders
% % • Sponsor, Mentor(s), Companies, Users, Markets, Fields, etc.
% % • Anyone that is affected by your proposed system
% % • Briefly describe needs and preferences of your clients and stakeholders


% \section*{Glossary}
% % Abbreviate all acronyms
% % • Amazon Web Services (AWS)

% \section*{References}



% \textbf{Your Project Title}

% \textbf{Final Report}

% Your Sponsor

% % \includegraphics[width=0.73491in,height=0.73491in]{media/image1.jpg}

% Sponsor logo (if any)

% \textbf{Mentor:}

% (Your mentor's name \& Organization)

% \textbf{Your Team's Name \& Logo}

% (Provide a list of team members)

% CptS 322 Software Engineering Principles I

% Semester Year

% \textbf{Note}: Recall that this writing assignment should follow these
% guidelines:

% Length = minimum of 5 pages text + appendixes as needed - though, this
% should be *MUCH* longer than 5 pages if you leverage all of your prior
% documents in full. There's no maximum page limit for this document, but
% be judicious in your screenshots that just fill space.\\
% \strut \\
% Sections that do not count to content for page limit:

% \begin{itemize}
% \item
%   Cover page
% \item
%   table of contents
% \item
%   pictures
% \item
%   tables
% \item
%   images
% \item
%   diagrams
% \end{itemize}

% As shown in the Syllabus, this assignment carries a weight of 20\% for
% the final course score.

% Posted as a single self‐contained file (no links to outside resources
% that should be directly in this document itself. A link to your git repo
% == good. A link to something that says ``see here for our detailed
% description''\ldots{} not as cool.)

% Posted as a PDF file.

% Typed single‐spaced.

% Typed with black text.

% Typed with \#11 font size.

% Typed using Arial font.

% Typed with one inch margins on sides, top and bottom.

% \textbf{\ul{Please erase this page in your final document.}}

\newpage
\tableofcontents
\newpage






\section{Introduction}\label{introduction}

% Describe your project background, motivation, and goals in detail.
\subsection{Abstract}
Many stores use point of sale (POS) systems to manage their inventory and sales.
These systems help store owners track sales and manage their inventory. These
are often touch-operated devices that also give the option of using a mouse
pointer or keyboard shortcuts. They consist of purpose-built machines running
proprietary, often custom-built, software.

Due to this proprietary nature, these systems are often expensive and are costly
and time consuming to update. This incurs higher maintenence costs on businesses
ityand a higher workload on employees due to the unreliabile of outdated software and hardware.

\subsection{Project Stakeholders}
Potential project stakeholders would include any physical retailer with the
desire to use this software. The goal is to design the software such that it can
be adapted to be used in any setting, offering a decent amount of customization. Although, the main
target audience, in the beginning, will be traditional storefront retailers.

\section{Team Members \& Bios}\label{team-members-bios}

\subsection{Calvin Tallent}
Calvin Tallent is a computer science student at Washington State University
interested in a wide variety of topics, from embedded automotive systems to
machine learning. He has experience in a wide variety of programming languages
and tools, including Python, Javascript, Rust, Java, C/C++, Kotlin, C\#, Git,
and general Linux administration.


\subsection{Owen Moore}
Owen Moore is a computer science student at Washington state University
interested in machine learning, unique computer archetectures and configurations,
and the intersection between music and computers. He has experience programing in C/C++, Java, Python.\par
\subsection{Giovanni Munoz}
Giovanni Munoz is a computer science student at Washington State University
interested in software applications and machine learning. Also, systems to solve real world problems.
He has experience in a variety of programming languages, including Python, Java, C/C++.

\par
% Include an entry in a narrative form for each of your team members (See alpha
% prototype for starting material \& update as required). The goal is to
% demonstrate the team's skills and project coverage. This is not just a pasted in
% resume, but a summary of your involvement in the project, and your technical
% interests. Include:


% Example:

% Joe Cougar is a computer science student interested in artificial
% intelligence, satellite development, and clock making. His prior
% projects have include smart homes, radio controlled dirigibles, and
% programming clocks. Joe's skills include C/C++, Python, RabbitMQ,
% Genetic Algorithms, and delinting. For this project his responsibilities
% include developing the Gamma Module, leading user experience feedback,
% and delivering sandwiches.

\section{Project Requirements
Specification}\label{project-requirements-specification}

Include the following:

\subsection{\texorpdfstring{Project Stakeholders
}{Project Stakeholders }}\label{project-stakeholders}

List your stakeholders, summarize their needs.

\subsection{Use Cases}\label{use-cases}

Describe the major use cases

\subsection{Functional Requirements}\label{functional-requirements}

Include functional requirements

\subsection{Non-Functional
Requirements}\label{non-functional-requirements}

Include non-functional requirements

\subsection{Standards}\label{standards}

Software engineering design is a process of devising a system,
component, or process to meet desired needs and specifications within
constraints. It is an iterative, creative, decision-making process in
which the basic sciences, mathematics, and engineering sciences are
applied to convert resources into solutions. Software engineering design
involves identifying opportunities, developing requirements, performing
analysis and synthesis, generating multiple solutions, evaluating
solutions against requirements, considering risks, and making
trade-offs, for the purpose of obtaining a high-quality solution under
the given circumstances.

Standards vary by project. Examples of possible standards include W3C
WCAG, GDPR, ECPA, FERPA, HIPAA, PCI DSS, IEEE 802, ACM SIGSOFT, IEEE
830, IEEE 1016, IEEE 829, and IEEE12207.

\section{\texorpdfstring{ Software Design - Solution
Approach}{ Software Design - Solution Approach}}\label{software-design---solution-approach}

This section should describe the final design of your software system.

Please review these sections and make the necessary updates to reveal
the changes in your project design since your revision.

\subsection{\texorpdfstring{Architecture Design
}{Architecture Design }}\label{architecture-design}

Include Section II from your Solution Approach report here. Provide the
block diagram of your architecture and give a brief description of it.

\subsubsection{Overview}\label{overview}

This section should describe the overall architecture of your software.
The architecture provides the top level design view of a system and
provides a basis for more detailed design work.

\subsubsection{Subsystem Decomposition}\label{subsystem-decomposition}

This section explains how you decomposed your system into subsystems.

\subsection{Data design}\label{data-design}

Describe all data structures (including the internal and temporary data
structures), and the database(s) created as part of the application.

\subsection{User Interface Design}\label{user-interface-design}

Provide a detailed description of user interface. The information in
this section should be accompanied with proper images of your software's
GUI.

The emphasis of this section should be on the design decisions of your
GUI rather than the final design. You will explain your final GUI in
detail in section VI (Description of Final Prototype). Please avoid
redundancy as much as possible in these two sections.

\subsection{Constraints}\label{constraints}

Software engineering design is a process of devising a system,
component, or process to meet desired needs and specifications within
constraints. It is an iterative, creative, decision-making process in
which the basic sciences, mathematics, and engineering sciences are
applied to convert resources into solutions. Software engineering design
involves identifying opportunities, developing requirements, performing
analysis and synthesis, generating multiple solutions, evaluating
solutions against requirements, considering risks, and making
trade-offs, for the purpose of obtaining a high-quality solution under
the given circumstances.

Constraints vary by project. Examples of possible constraints include
aesthetics, codes, constructability, cost, ergonomics, extensibility,
functionality, interoperability, legal considerations, maintainability,
manufacturability, marketability, policy, regulations, schedule,
sustainability, or usability.

\section{Test Case Specifications and
Results}\label{test-case-specifications-and-results}

\subsection{Testing Overview}\label{testing-overview}

Include a summary of your test objectives and test plans.

Describe the overall approach to testing and provide the overall flow of
the testing process. An example is provided in an Appendix.

Are you using Continuous Integration (CI) and/or Continuous Delivery
(CD) in your testing? What tools are you using, what kinds of testing do
they provide? Unit tests, integration, system, user interface,
speed/non-functional requirement testing, user testing, etc.

In all of these major categories, include at least some material about
your implementation, how it is currently executing, and/or what future
work would be needed to ensure it happens for this project

\begin{itemize}
\item
  Unit testing
\item
  Integration testing
\item
  System testing
\item
  Functional testing
\item
  Performance testing
\item
  Standards and constraints testing
\item
  User acceptance testing
\end{itemize}

\subsection{\texorpdfstring{ Environment
Requirements}{ Environment Requirements}}\label{environment-requirements}

Specify both the necessary and desired properties of the test
environment. The specification should contain the physical
characteristics of the facilities, including the hardware, the
communications and system software, the mode of usage (for example,
stand-alone), and any other software or supplies needed to support the
test. Identify special test tools needed.

\subsection{Test Results}\label{test-results}

Include your prototype test results from your ``Test Case Specifications
and Results'' document as updated with your current project status.

\subsection{}\label{section}

\section{Projects and Tools used}\label{projects-and-tools-used}

Include a summary of the libraries, frameworks, and tools you used to
implement your project. For example, you could have used some web
frameworks, a database, a network message passing tool, graphics
generation libraries, and various operating system platforms. Please
list these out with a short one sentence note about what it was used to
build/support in your project.

\begin{longtable}{p{0.25\linewidth}p{0.75\linewidth}}
\toprule
Tool/library/framework & Quick note on what it was for \\
\midrule
Bootstrap & Generating layout and visual rendering on web interface \\
RabbitMQ & Intercommunication between our modules \\
\bottomrule
\end{longtable}

As a quick survey, let me know what languages you wrote some of the
project in. This is anything you wrote yourself, not just used in
libraries. This includes both programming languages and markup
languages.

\begin{longtable}{p{0.25\linewidth}p{0.25\linewidth}p{0.25\linewidth}p{0.25\linewidth}}
\toprule
\multicolumn{4}{c}{Languages Used in Project} \\
\midrule
C & C++ & JavaScript & Erlang \\
Java & Python & PERL & Go \\
HTML5 & LaTeX & BASH shell & \\
\bottomrule
\end{longtable}

\section{Description of Final
Prototype}\label{description-of-final-prototype}

In this section provide a \textbf{\ul{detailed}} description of your
final prototype.

(If applicable) Include a brief user manual for your final software
where you provide step by step instructions for using your system.
Explain the major use case scenarios. You may include screenshots of the
user interface.

\ul{Describe your final prototype implementation. Please format this
section according to what you think is the best way to describe your
prototype. The following is just a suggestion.}

I recommend to include plenty of images and pictures of the following
where appropriate:

- any diagrams/figures that visualize various features of your
prototype;

- the screenshots of your user interfaces;

- the screenshots of your test programs;

- pictures of your team testing and debugging the devices, programs,
etc.

A well-thought and clear diagram is better than long and descriptive
text.

If your document starts to be very long due to screenshots and diagrams,
please put at least some of them into an appendix to this document.

\section{Social Responsibility and Broader
Impacts}\label{social-responsibility-and-broader-impacts}

Social responsibility: identify informed judgements that you made for
this project based on legal and ethical principles. Discuss if those
judgements are in line with your responsibilities as a computing /
software / cyber engineer. If not, what prompted you to make judgements
that are contradictory to your professional responsibilities?

Broader impacts: highlight how your work aligns with the WSU EECS's
goals of benefiting society and broadening participation in STEM.
Describe not only how your project addressed a real-world problem
(briefly state the problem) but also emphasized accessibility,
inclusivity, and/or educational outreach. If you engaged with local
communities through user testing and feedback sessions, mention that. If
you collaborated with multiple teams from various departments or groups,
mention that. If you documented your project and released it freely to
support open-source missions, mention that. If your project outcomes
contribute to technological innovation and provide a foundation for
future student-led research, community-based applications, mention that.

\section{Product Delivery Status}\label{product-delivery-status}

When was/will the project be delivered to your client? Was it
demonstrated and who did you hand it off to?

Also include project location information here. This is for both the
instructor and your client as future reference. This should include:

\begin{enumerate}
\def\labelenumi{\arabic{enumi})}
\item
  Source repositories
\item
  Equipment storage location - where did you leave any client or course
  materials
\item
  Any other materials needed to rebuild your project from the ground up
\end{enumerate}

Ensure that either here or in your detailed description you include any
instructions or where the instructions are to install and setup your
project so future users know where to start.

\section{Conclusions and Future Work}\label{conclusions-and-future-work}

\subsection{Limitations and
Recommendations}\label{limitations-and-recommendations}

Include a brief description of the limitations of your current prototype
that you are aware of, and discuss possible solution approaches to
overcome those limitations.

Please keep the discussion brief. 1 page text should be sufficient for
this section.

Note that there might be overlap between the Limitations and
Recommendations section and the Future Work section. ``Future Work''
should discuss the possible ways to extend the project, where as
``Limitations and Recommendations'' should be more detailed and should
discuss the problems and missing features in the final prototype and
should describe the possible solution approaches.

\subsection{Future Work}\label{future-work}

Include a conclusion and discuss the future work. Future work should
include discussion on how to extend this project. For those of you who
are self or EECS sponsored, discuss commercialization possibilities.

\section{Acknowledgements}\label{acknowledgements}

Thank the people who have contributed to your project. Also thank to
your sponsors.

\section{Glossary}\label{glossary}

\begin{quote}
Define technical terms, acronyms, and anything else the reader might
need to look up as they go through your document
\end{quote}

\section{References}\label{references}

Cite all your references here. For the papers you cite give the authors,
the title of the article, the journal name, journal volume number, date
of publication and inclusive page numbers. Giving only the URL for the
journal is not appropriate.

Use either Chicago, IEEE, or ACM style citations

-\/- Note: you can find many articles on scholar.google.com, which
includes a link for each article's citation in various formats.

% \includegraphics[width=4.54688in,height=1.39175in]{media/image2.png}

\section{Appendix A -- Team
Information}\label{appendix-a-team-information}

List your team members here and provide a team photo.

\section{Appendix B - Example Testing Strategy
Reporting}\label{appendix-b---example-testing-strategy-reporting}

\begin{enumerate}
\def\labelenumi{\arabic{enumi})}
\item
  Identify the requirements being tested
\item
  Either link to available online test results and/or take screenshots
  of the various system testing results
\item
  User testing can be demonstrated via survey results or quotes and a
  discussion of the feedback received
\end{enumerate}

\section{Appendix C - Project
Management}\label{appendix-c---project-management}

Describe your team's weekly schedule, i.e.,

\begin{itemize}
\item
  weekly meetings with the instructor/ mentor
\item
  weekly meetings only with the team members
\item
  other meetings and project related team activities.
\end{itemize}

Explain the purpose of each of the above activities and describe the
routine agenda for each.

Please comment on which team activities/meetings were the most
beneficial to your team.

Please include any planning documents you may have used. Examples could
include:

\begin{itemize}
\item
  Gantt charts
\item
  GitHub projects - Screenshots \& a figure description
\item
  GitHub issues - Screenshots \& a figure description
\item
  Notes on team tools used:

  \begin{itemize}
  \item
    Email
  \item
    SMS/IM
  \item
    Slack
  \item
    appear.in or other video conference tools
  \item
    Trello
  \item
    etc.
  \end{itemize}
\end{itemize}
